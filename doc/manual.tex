\documentclass[paper=a4, fontsize=11pt]{article} % A4 paper and 11pt font size
\usepackage[utf8]{inputenc}
\usepackage[T1]{fontenc} % Use 8-bit encoding that has 256 glyphs
\usepackage{fourier} % Use the Adobe Utopia font for the document - comment this line to return to the LaTeX default
\usepackage[ngerman,british,UKenglish,USenglish,american]{babel}
\usepackage[utf8]{inputenc}
\usepackage{amsmath,amsfonts,amsthm} % Math packages
\usepackage{hyperref}
\usepackage{fancyhdr}
\usepackage{graphicx}
\usepackage{listings}
\usepackage{todonotes}
\usepackage{titlesec}
\usepackage{xcolor}
\usepackage{ulem}
\definecolor{grey}{rgb}{0.4, 0.4, 0.4}

%\usepackage{sectsty} % Allows customizing section commands
%\allsectionsfont{\centering \normalfont\scshape} % Make all sections centered, the default font and small caps

\usepackage{fancyhdr} % Custom headers and footers
\pagestyle{fancyplain} % Makes all pages in the document conform to the custom headers and footers
\fancyhead[R]{} % No page header - if you want one, create it in the same way as the footers below
\fancyfoot[L]{} % Empty left footer
\fancyfoot[C]{\thepage} % Page numbering for center footer
\fancyfoot[R]{} % Empty right footer
\renewcommand{\headrulewidth}{0pt} % Remove header underlines
\renewcommand{\footrulewidth}{0pt} % Remove footer underlines
\setlength{\headheight}{13.6pt} % Customize the height of the header

%farbige Hyperlinks
%\definecolor{refcolor}{rgb}{0,.2,.4}
%schwarze Hyperlinks
\definecolor{refcolor}{rgb}{0,0,0}
%Hyperref Color
\hypersetup{pdftex=true, colorlinks=true, breaklinks=true, linkcolor=refcolor, menucolor=refcolor, pagecolor=refcolor, citecolor=refcolor, urlcolor=blue}

\numberwithin{equation}{section} % Number equations within sections (i.e. 1.1, 1.2, 2.1, 2.2 instead of 1, 2, 3, 4)
\numberwithin{figure}{section} % Number figures within sections (i.e. 1.1, 1.2, 2.1, 2.2 instead of 1, 2, 3, 4)
\numberwithin{table}{section} % Number tables within sections (i.e. 1.1, 1.2, 2.1, 2.2 instead of 1, 2, 3, 4)

\setlength\parindent{0pt} % Removes all indentation from paragraphs - comment this line for an assignment with lots of text
% Code Listing Style
\definecolor{darkblue}{rgb}{0,0,.6}
\definecolor{darkgreen}{rgb}{0,0.5,0}
\definecolor{darkred}{rgb}{0.5,0,0}
\lstset{
	basicstyle=\ttfamily,
	commentstyle=\color{darkgreen},
	keywordstyle=\color{darkblue}\fontseries{sb}\fontshape{n}\selectfont,
	stringstyle=\color{darkred},
%	identifierstyle=\color{darkgreen},
%    moredelim=[is][\underbar]{_}{_},
	breaklines=true,
	tabsize=2,
%	xleftmargin=-1mm,
%	xrightmargin=3mm,
%	aboveskip=\smallskipamount,
%	belowskip=\smallskipamount,
	numbers=none,
	frame=none,
	showstringspaces=false,
	captionpos=t,
%	framexbottommargin=3pt,
%	framextopmargin=3pt,
	% Umlaute
	literate=%
		{Ö}{{\"O}}1
		{Ä}{{\"A}}1
		{Ü}{{\"U}}1
		{ß}{{\ss}}2
		{ü}{{\"u}}1
		{ä}{{\"a}}1
		{ö}{{\"o}}1
}

% BibLaTeX
%\usepackage[style=authoryear]{biblatex}
%\usepackage[style=numeric]{biblatex}
\usepackage[style=alphabetic]{biblatex}

\AtEveryBibitem{\clearlist{language}} % clears language
\AtEveryBibitem{\clearfield{note}}    % clears notes
\AtEveryBibitem{\clearfield{doi}} % clears doi
\AtEveryBibitem{\clearfield{isbn}} % clears doi
\AtEveryBibitem{\clearfield{issn}} % clears doi
\addbibresource{bibliography.bib} % Syntax for version >= 1.2

%----------------------------------------------------------------------------------------
%	TITLE SECTION
%----------------------------------------------------------------------------------------

\newcommand{\horrule}[1]{\rule{\linewidth}{#1}} % Create horizontal rule command with 1 argument of height

\title{	
\normalfont \normalsize 
\textsc{\includegraphics[width=0.6\textwidth]{pictures/logo} \\ [5pt] Arbeitsgruppe Datenbanken und Informationssysteme \\ [20pt] \includegraphics[width=0.15\textwidth]{pictures/DBIS_Logo_rgb_web.png}} \\ [10pt] % Your university, school and/or department name(s)
\horrule{0.5pt} \\[0.4cm] % Thin top horizontal rule
\huge Spatial Databases:\\ Project Documentation (orphaned) \\ [0.15cm] % The assignment title
\normalsize \textsc{Setup-Guide and Documentation for Spatial-Weather-Project} \\ [0.4cm]
\horrule{2pt} \\[0.5cm] % Thick bottom horizontal rule
}

\author{Christian Wirth (4498611) \\ Miriam Seel (dropped out)}

\date{\today}

\begin{document}
\begin{titlepage}
\pagenumbering{Roman}
\maketitle
\thispagestyle{empty}
\end{titlepage}

\setcounter{page}{1}
\addcontentsline{toc}{section}{\protect\numberline{}{Table of Contents}}
\tableofcontents

\newpage
\pagenumbering{arabic}
\pagestyle{fancy}
\setcounter{page}{1}
\section{Task}

The goal of the project is to work with weather data and OpenStreetMap (OSM). 
The steps to be taken are as follows \footnote{Task as stated in the Project-description by Daniel Kressner}: 

\begin{itemize}
      \item Find data sources for: 
      \begin{itemize}
         \item Medium-range (3–7 days) and short-range (12–48 hours) weather forecasts
         \item Current and historical weather data from weather stations
         \item OpenStreetMap data for districts and cities
         
      \end{itemize}
      \item Model your data in an ER and store the data in PostGIS.
      \item Overlay OSM areas with forecast and historical weather data.
      \item Visualize the probability of forecasts compared to historical weather data on a map.
   \end{itemize}


\section{Project}
\label{sec:Umsetzung}

\subsection{Responsibilities}

Christian:
	\begin{itemize}
    	\item PostGIS-server set-up \& configuration
    	\item Download \& importing OSM-data
   		\item Backend-logic
    	\item Frontend
    \end{itemize}
    
Miriam:
	\begin{itemize}
    	\item Parsing \& importing weather-data
    	\item Schema \& data-model
    	\item Queries \& query-optimisation
    	\item Documentation
    \end{itemize}

\newpage
\subsection{Environment}
\begin{itemize}
	\item virtual server (orphaned)
    \item PostGIS (orphaned)
    \item Extensions activated (orphaned)
    \begin{itemize}
    	\item Enable PostGIS (includes raster) -> CREATE EXTENSION postgis;
   		\item Enable Topology -> CREATE EXTENSION postgis\_topology;
    	\item Fuzzy matching -> CREATE EXTENSION fuzzystrmatch;\\
    	Fuzzy string matching was activated in order to implement a fuzzy string search for locations.
    \end{itemize}
    \item Frontend: Leaflet
    \item Backend
     \begin{itemize}
    	\item \textbf{Option A:} Express for Nodejs\\ [0.1cm]
    	Our approach for the Backend follows a guide on Boomphisto' Blog
    	\footnote{http://boomphisto.blogspot.de/2011/07/nodejs-express-leaflet-postgis-awesome.html (last access: 28.02.2015)}, which explains the set-up of a Node-server with Express, a framework providing controllers and libraries for PostGIS as well as Leaflet.
    	The article claims, that Express provides libraries in JavaScript to access the PostGIS-server and returns GeoJSON, which can be directly forwarded.\\ Sadly this project had to be aborted, before we were able to verify this claim and evaluate how much additional work has be done to get it up and running.\\
    	If you follow the guide mentioned above you can start the nodejs-server, but won't see anything apart from a blank page, since we abandoned the project during the set-up of the ejs-based user interface\footnote{http://codeforgeek.com/2014/06/express-nodejs-tutorial/ (last access: 28.02.2015)} for express 
   		\item \textbf{Option B:} Leaflet with yet undefined backend logic in python.\\ [0.1cm]
   		\textbf {Accessible by:}
   		\begin{lstlisting}
   		spatialdb_project/weatherdb/index.html
   		\end{lstlisting}
   		This was our fall-back-approach, in case we would run into essential problems with Option A, since we knew, that Option B already worked for other teams.
    \end{itemize}
\end{itemize}

\newpage
\subsection{Saving weather-data}
Openweathermap offers weather forecasts for free as XML or json. We decided to download the data as json-files, because the size of data is much smaller (http://openweathermap.org/forecast). This Data has a timestamp in Unix-Time, so it will be necessary to convert that format to date-format. 
\begin{lstlisting}
{"cod":"200","message":0.0139,
"city":{
	"id":2950159,
    "name":"Berlin",
	"coord":{"lon":13.41053,"lat":52.524368},
	"country":"DE","population":0,
	"sys":{"population":0}},
"cnt":1,
"list":[{
	"dt":1417860000, //unix-time
	"temp":{
		"day":4.46,
		"min":3.79,
		"max":4.46,
		"night":3.79,
		"eve":4.46,
		"morn":4.46},
	"pressure":1023.77,
	"humidity":100,
	"weather":[{"id":600,"main":"Snow","description":"light snow",
    		"icon":"13d"}],"speed":2.17,"deg":226,"clouds":88,
   	 		"snow":0.25}]}
\end{lstlisting}

\begin{center}
    \begin{tabular}{ | l | p{5cm} |}
    \hline
    \textbf{Parameter} & \textbf{Description} \\ \hline
    city.id & City identification \\ \hline
    city.name & City name \\ \hline
    city.country & Country (GB, JP etc.) \\ \hline
    
    coord.lat & City geo location, lat \\ \hline
    coord.lon & City geo location, lon \\ \hline
    
    cnt & Number of lines returned by this API call \\ \hline
    dt & Data receiving time, unix time, GMT \\ \hline
    temp.day & Day temperature \\ \hline
    temp.min & Min daily temperature \\ \hline
    temp.max & Max daily temperature \\ \hline
    temp.night & Night temperature \\ \hline
    temp.eve & Evening temperature \\ \hline
    temp.morn & Morning temperature \\ \hline
    humidity & Humidity \\ \hline
    pressure & Atmospheric pressure \\ \hline
    wind.speed & Wind speed, mps \\ \hline
    wind.deg & Wind direction, degrees (meteorological) \\ \hline
    wind.gust & Wind gust, mps \\ \hline
    clouds.all & Cloudiness \\ \hline
    weather.id & Weather condition id \\ \hline
    weather.main & Group of weather parameters (Rain, Snow, Extreme etc.) \\ \hline
    weather.description & Weather condition within the group \\ \hline
    weather.icon & Weather icon id \\ \hline
    rain & Precipitation volume for last 3 hours, mm \\ \hline
    snow & Snow volume for last 3 hours, mm \\ \hline
    \end{tabular}
\end{center}

On the owm-HP a fix json-file exists with all weatherstations (cities) where weather is measured. It is structured as follows:\newline
 \begin{lstlisting}
{"_id":2947416,"name":"Bochum","country":"DE",
	"coord":{"lon":7.21667,"lat":51.48333}}
{"_id":567322,"name":"Chvizhepse","country":"RU",
	"coord":{"lon":40.085278,"lat":43.631111}}
\end{lstlisting}
At first we created a new file with all the german cities with groovy. The number of rows decreased from 209579 to 28783.
\newpage
\definecolor{myGrey}{gray}{0.9}

\begin{lstlisting}[language=java, 		% Setzt die Sprache
	basicstyle=\scriptsize\ttfamily, 	% Setzt den Standardstil
	keywordstyle=\color{red}\bfseries,	% Setzt den Stil für Schlüsselwörter
	identifierstyle=\color{blue},		% Identifier bekommen keine gesonderte formatierung
	commentstyle=\color{DarkGreen},		% Stil für Kommentare
	stringstyle=\ttfamily, 				% Stil für Strings (gekennzeichnet mit "String")
	breaklines=true, 					% Zeilen werden umgebrochen
	numbers=left, 						% Zeilennummern links
	numberstyle=\tiny, 					% Stil für die Seitennummern
	frame=single, 						% Rahmen
	backgroundcolor=\color{myGrey}, 	% Hintergrundfarbe
	caption={groovy-code for cityfilter},	% Caption
	tabsize=2							% Größe der Tabulatoren
	]
class FilterCityList {

	public static void filterCitiesFileByCountryDe(def fileName){

		def file = new File(fileName)
		StringWriter writer = new StringWriter()
		def fw = new FileWriter("/germanCities.txt")
		file.filterLine(writer) { line ->
			line.contains("\"country\":\"DE\"")
		}
		println writer.toString()
		fw.write(writer.toString())
		fw.close()
	}
}


class FilterCityListTest {

	@Test
	public void testFilterCitiesFileByCountryDe() {
		FilterCityList.filterCitiesFileByCountryDe("/city.list.json")
		assertTrue(true);
	}

}
\end{lstlisting}

With this new file we created all the URLs to get weather-forecasts. This list of URLs is used for the daily download.

\subsection{Schema}

\textbf{weatherStation}(cityId, cityName, cityCountry, longitude, latitude)\newline
\textbf{forecastData}(weatherstationId, dataTime, tempDay, tempMin, tempMax, tempNight, tempEve, tempMorn, humidity, pressure, windSpeed, windDirection, windGust, cloudiness, weatherCondition, weatherDecription, weatherIcon, rain, snow, insertdate)\newline
\textbf{weatherData}(weatherstationId, currentFlag, dataTime, tempDay, tempMin, tempMax, tempNight, tempEve, tempMorn, humidity, pressure, windSpeed, windDirection, windGust, cloudiness, weatherCondition, weatherDecription, weatherIcon, rain, snow, insertdate)\\
\ldots

\section{Appendix}
\subsection{Link to Repository}
\begin{lstlisting}
https://github.com/CrusaderW/spatialdb_project
\end{lstlisting}

\end{document}